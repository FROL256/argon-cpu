\title{Argon CPU manual}
\author{
        Frolov Vladimir \\
        KIAM RAS, Moscow
}
\date{\today}

\documentclass[12pt]{article}

\begin{document}
\maketitle

\begin{abstract}
This is paper in an unformal introduction to Argon CPU architecture mainy needed for compiler developers, students, guys engaged in Argon or other chip design project and, of cource, the end users. 

\end{abstract}


\section{The target audience}
So, if fact any guys who may be interested in developing or using a custom tiny 32 bit RISC processor, for example, routed in FPGA. If you are interested in a chip design via VHDL, or want to know more about CPU architecture, you are welcome.


\section{Motivation}

What reason for did we make an own processor? There are several of them:

\begin{enumerate}
\item The study case. If you want to understand the CPU architecrure, best way to do it is to make your own CPU. Due to that simplicity was our main goal.

\item Minimalism and simplicity. The most of current RISC architecrures are far from being minimal and their ISA is abundant. Even MIPS. We wanted to create a really small, 'simply-stupid' chip with a custom ISA and full functionality of common 32 bit RISC microprocessor like MIPS.

\item Fun. Making CPU is a big fun, so, join us or steal our ideas and make your own project!

\end{enumerate}

\section{Introduction}

\section{ISA}




\end{document}
This is never printed
  